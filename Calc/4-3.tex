%%%%%%%%%%%%%%%%%%%%%%%%%%%%%%%%%%%%%%%%%%%%%%%%%%%%%%%%%%%%%%%%%%%%%%%%%%%%%%%%%%%%
% Do not alter this block (unless you're familiar with LaTeX
\documentclass{article}[12pt]
\usepackage[margin=1in]{geometry} 
\usepackage{amsmath,amsthm,amssymb,amsfonts, fancyhdr, color, comment, graphicx, environ, tikz}

\usepackage{mdframed}
\usepackage[shortlabels]{enumitem}
\usepackage{indentfirst}
\usepackage{hyperref}
\hypersetup{
    colorlinks=true,
    linkcolor=blue,
    filecolor=magenta,      
    urlcolor=blue,
}


\pagestyle{fancy}

\theoremstyle{definition}
\newtheorem{definition}{Definition}[section]
\theoremstyle{example}
\newtheorem{example}{Example}[section]
\theoremstyle{theorem}
\newtheorem{theorem}{Theorem}[section]
\newtheorem{remark}{Remark}[section]
\newtheorem{question}{Question}[section]
\newtheorem{rules}{Rule}[section]
% prevent line break in inline mode
\binoppenalty=\maxdimen
\relpenalty=\maxdimen

%%%%%%%%%%%%%%%%%%%%%%%%%%%%%%%%%%%%%%%%%%%%%
%Fill in the appropriate information below

\rhead{Calculus} 
\chead{4-3 Initial Value Problems}
\begin{document}
\section{Revisiting the last problems on the homework:}
\begin{enumerate}
        \item Find $f(x)$ if $f'(x)=2x$ and $f(1)=2$.
        \vspace{1cm}
        \item Find $g(x)$ if $g'(x)=2x$ and $g(0)=2$. 
        \vspace{1cm}
        \item Find $h(x)$ if $h'(x)=\sin(x)$ $h(0)=5$.
        \vspace{1cm}
\end{enumerate}
\section{So What is happening?}
When we are given an \textbf{initial condition} we can turn our ``$+c$'' into an actual number and have a function we care about and know. 
\section{Why should I care?}
\begin{example}
Suppose the velocity function for a particle is $v(t)=\frac{1}{10}t^2+2$ feet/seconds.
\begin{enumerate}
\item Find a position function for the particle. It's okay if you have an unknown in your equation. 
\vspace{2cm}
\item Suppose that the position of the particle is 10 feet at $t=3$ seconds. Solve for the unknown constant in part (1) and re-write the position function with this new information. 
\vspace{2cm}
\item Suppose now that the position of the particle is 2 feet at $t=1$ seconds. 
\vspace{2cm} 
\end{enumerate} 
\end{example}
\begin{example}
Suppose that the acceleration of a car from a stop is given by $a(t)=\sqrt{t}$ feet/s$^2$. 
\begin{enumerate}
\item Find the velocity function $v(t)$ for this car. What are the units of this function?
\vspace{2cm}
\item The fact that the car starts means that at $t=0$, the velocity is zero, that is, $v(0)=0$. Use this to figure out what the unknown is in the formula for $v(t)$. 
\vspace{2cm}
\item Find the velocity of the car after $t=4$ seconds. 
\vspace{1cm}
\end{enumerate}
\end{example}
\begin{example}
Acceleration due to Earth's gravity is $9.8 m/s^2$ (down). 
\begin{enumerate}
\item The acceleration function $a(t)$ is often written as $a(t)=-9.8$. Why is it negative?
\vspace{2cm}
\item If an object is thrown upwards with an initial velocity of $10m/s$, find a function which gives the velocity of the object after t seconds. Note that the word initial is used to mean at time $t = 0$.
\vspace{3cm}
\item If this same object is thrown from the top of a $150m$ building, that is, $d(0) = 150$, find a function which gives the height of the object after $t$
seconds.
\vspace{3cm}
\end{enumerate}
\end{example}
\begin{example}
A tree nursery usually sells a certain shrub after 6 years of growth and shaping. The growth rate is approximated by 
\[h'(t)=.5t+2\] where $t$ is the time in years and $h$ is the height in inches. The seedlings are 5 inches tall when planted. 
\begin{enumerate}
\item Find the height after $t$ years. 
\vspace{2cm}
\item How tall are the shrubs when they are sold?

\end{enumerate}
\end{example}
\newpage
\section{Practice/ Homework}
\begin{enumerate}
\item Suppose Galileo drops a cannonbal off the Leaning Tower of Pisa (60 meters off the ground). 
\begin{enumerate}
\item Write down the acceleration function, $a(t)$, for the cannonball. 
\item Write a function which gives the velocity of the cannonball after $t$ seconds. Note that 	``dropped" means the initial velocity of the cannonball is 0 m/s. 
\item Find a function for the height of the cannonball (off the ground) after $t$ seconds. 
\item When does the cannonball hit the ground?
\item What is the speed of the cannonball when it hits the ground?
\end{enumerate}
\item One of the lunar astronauts dropped a wrench from the top of the space module (12 meters above the surface of the moon). How long did it take for the wrench to hit the ground? (The acceleration due to gravity on the moon is $1.622m/s^2$
down.)
\item A car is traveling at a rate of $88$ ft/sec when the brakes are applied. (ie $v(0)=88$). The car begins decelerating at a constant rate of $15ft/sec^2$. 
\begin{enumerate}
\item How many seconds elapse before the car stops?
\item How far does the car travel during that time?
\end{enumerate} 
\item A water balloon launcher has been installed on the roof of a building. The vertical velocity (in feet/sec) of the water balloon at time $t$ seconds after launch is given by \[v(t)=-32t+40,\] with $v>0$ corresponding to upward motion. 
\begin{enumerate}
\item If the roof of the building is 30 feet above the ground, find an expression for the height of the water balloon above the ground at any time $t$. 
\end{enumerate}
\item A stone thrown upwards from the top of a 320 ft cliff at 128 ft/sec eventually falls to the beach below. 
\begin{enumerate}
\item How long does the stone take to reach its highest point?
\item What is the maximum height?
\item How long before the stone hits the beach?
\item What is the velocity of the stone when it hits the ground?
\end{enumerate}
\item My cat (Nox) knocked my mug off the counter. The counter is 3 feet high, and the mug fell (and broke) after 2 heart stopping seconds.  Note that acceleration due to gravity is approximately 32 $ft/s^2$. 
\begin{enumerate}
\item What was the speed that Nox hit the mug with? 
\item Find an equation that models the height of the mug $t$ seconds after Nox hit it. 
\end{enumerate}
\item A faucet is turned on at 8am and water flows in at a rate of $r(t)=8\sqrt{t}$, where $t$ is given in hours after 8am and $f(t)$ is given in gallons/hour. 
\begin{enumerate}
\item I want to fill an empty 10 gallon tank. How long do I need to wait until the 10 gallon tank is filled? 
\end{enumerate}
\end{enumerate}
\end{document}