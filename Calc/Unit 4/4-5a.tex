%%%%%%%%%%%%%%%%%%%%%%%%%%%%%%%%%%%%%%%%%%%%%%%%%%%%%%%%%%%%%%%%%%%%%%%%%%%%%%%%%%%%
% Do not alter this block (unless you're familiar with LaTeX
\documentclass{article}[12pt]
\usepackage[margin=1in]{geometry} 
\usepackage{amsmath,amsthm,amssymb,amsfonts, fancyhdr, color, comment, graphicx, environ, tikz}
 \usepackage{ulem}
\usepackage{mdframed}
\usepackage[shortlabels]{enumitem}
\usepackage{indentfirst}
\usepackage{hyperref}
\hypersetup{
    colorlinks=true,
    linkcolor=blue,
    filecolor=magenta,      
    urlcolor=blue,
}


\pagestyle{fancy}

\theoremstyle{definition}
\newtheorem{definition}{Definition}[section]
\theoremstyle{example}
\newtheorem{example}{Example}[section]
\theoremstyle{theorem}
\newtheorem{theorem}{Theorem}[section]
\newtheorem{remark}{Remark}[section]
\newtheorem{question}{Question}[section]
\newtheorem{rules}{Rule}[section]
% prevent line break in inline mode
\binoppenalty=\maxdimen
\relpenalty=\maxdimen

%%%%%%%%%%%%%%%%%%%%%%%%%%%%%%%%%%%%%%%%%%%%%
%Fill in the appropriate information below

\rhead{Calculus} 
\chead{4-5b $u$-Substitution (again)}
\begin{document}
\section*{Motivation}
The two goals of $u$-substitution are to make a hard integral into an easier one and to undo the chain rule.  \\
\section{First Step:}
First let's look at functions where we found the derivative using the chain rule and identify the $g(x)$ (which will be later used at the $u$). \\
For each function below, it looks like $f'(g(x))\cdot g'(x)$, or $g'(x)\cdot f'(g(x))$, for some $f$ and $g(x)$. Identify $g(x)$. 
\begin{enumerate}
\item $4x(x^2+4)$
\item $(4w^3-6w)e^{w^4-3w^2+9}$
\item $50(6t^2-\sin(t))(2t^3+\cos(t))^{49}$
 (this one might be hard)
\end{enumerate}
\section{Second Step:}
Now that we have identified $g(x)$, let's let $u=g(x)$, and for each function below, replace $u$ with $g(x)$. At this point, every function will contain 2 variables,$u$ and the original variable.  
\begin{enumerate}
\item $4x(x^2+4)$
\vspace{1.5cm}
\item $(4w^3-6w)e^{w^4-3w^2+9}$
\vspace{1.5cm}
\item $50(6t^2-\sin(t))(2t^3+\cos(t))^{49}$
\vspace{1.5cm}

\end{enumerate}
\section{Third Step:}
If we consider the equation $u=g(x)$, and then took the derivative of both sides, we would have $u'=g'(x)$, also written and expressed as $du=g(x)dx$. So now, for each $u$ equation, find the derivative of both sides. 
\begin{enumerate}
\item 
\item 
\item 

\end{enumerate}
\section{Fourth Step:}
Now, we should be able to see $g'(x)$ in all of our original functions. Recall that the goal was to learn a new integration technique. So let's now rewrite every original function as an integral in terms of $u$: 
\begin{enumerate}
\item $\int4x(x^2+4)dx$
\vspace{1.5cm}
\item $\int(4w^3-6w)e^{w^4-3w^2+9}dw$
\vspace{1.5cm}
\item $\int50(6t^2-\sin(t))(2t^3+\cos(t))^{49}dt$
\vspace{1.5cm}

\end{enumerate}
\section{Fifth Step:}
Each of these integrals (in terms of $u$) is objectively easier to solve than the original function. Let's solve them:
\begin{enumerate}
\item 
\item 
\item 
\end{enumerate}
\section{Sixth Step:}
Now let's rewrite all of our antiderivatives so that they contain the original variable: 
\begin{enumerate}
\item 
\item 

\item 
\end{enumerate}
\section{More Practice (not homework)}
\begin{enumerate}
\item $\int(2x+2)e^{x^2+2x+3}dx$
\item $\int(2x+5)(x^2+5x)^7dx$
\item $\int xe^{x^2+1}dx$
\item $\int \frac{x^2+1}{x^3+3x}dx$
\item $\int (3-x)^{10}dx$
\item $\int 5e^{5x+2}dx$
\item $\int 8x^7(1+x^8)^31dx$
\item $\int x^7(1+x^8)^31dx$
\end{enumerate}
\end{document}